%%%%%%%%%%%%%%%%%%%%%%%%%%%%%%%%%%%%%%%%%%%%%%%%%%%%%
%												    %
%	LCOM PROJECT PROPOSAL				    		%
%												    %
%	Novembro 2015								    %
%												    %
%	Angela Cardodo and Ilona Generalova				%
%   											    %	
%%%%%%%%%%%%%%%%%%%%%%%%%%%%%%%%%%%%%%%%%%%%%%%%%%%%%

\documentclass[11pt,a4paper,reqno]{report}
\linespread{1.2}


\usepackage{rotating}
\usepackage{tikz}
\usepackage[active]{srcltx}    
\usepackage{graphicx}
\usepackage{amsthm,amsfonts,amsmath,amssymb,indentfirst,mathrsfs,amscd}
\usepackage[mathscr]{eucal}
\usepackage{tensor}
\usepackage[utf8x]{inputenc}
%\usepackage[portuges]{babel}
\usepackage[T1]{fontenc}
\usepackage{enumitem}
\setlist{nolistsep}
\usepackage{comment} 
\usepackage{tikz}
\usepackage[numbers,square, comma, sort&compress]{natbib}
\usepackage[nottoc,numbib]{tocbibind}
%\numberwithin{figure}{section}
\numberwithin{equation}{section}
\usepackage{scalefnt}
\usepackage[top=2cm, bottom=2.5cm, left=2cm, right=2cm]{geometry}
\usepackage{MnSymbol}
%\usepackage[pdfpagelabels,pagebackref,hypertexnames=true,plainpages=false,naturalnames]{hyperref}
\usepackage[naturalnames]{hyperref}
\usepackage{enumitem}
\usepackage{titling}
\newcommand{\subtitle}[1]{%
	\posttitle{%
	\par\end{center}
	\begin{center}\large#1\end{center}
	\vskip0.5em}%
}
\newcommand{\HRule}{\rule{\linewidth}{0.5mm}}
\usepackage{caption}
\usepackage{etoolbox}% http://ctan.org/pkg/etoolbox
\usepackage{complexity}

\usepackage[official]{eurosym}

\def\Cpp{C\raisebox{0.5ex}{\tiny\textbf{++}}}

\makeatletter
\def\@makechapterhead#1{%
  \vspace*{2\p@}%
  {\parindent \z@ \raggedright \normalfont
    \ifnum \c@secnumdepth >\m@ne
%        \huge\bfseries \@chapapp\space \thechapter
        \par\nobreak
        \vskip 5\p@
    \fi
    \interlinepenalty\@M
    \Huge \bfseries \thechapter\space #1\par\nobreak
    \vskip 20\p@
  }}
\def\@makeschapterhead#1{%
  \vspace*{2\p@}%
  {\parindent \z@ \raggedright
    \normalfont
    \interlinepenalty\@M
    \Huge \bfseries  #1\par\nobreak
%    \vskip 5\p@
  }}
\makeatother

\usepackage[toc,page]{appendix}


%\addto\captionsportuges{%
%  \renewcommand\appendixname{Anexo}
%  \renewcommand\appendixpagename{Anexos}
%  \renewcommand\appendixtocname{Anexos}
%  \renewcommand\abstractname{\huge Sumário}  
%}

\usepackage{verbatim}
\usepackage{color}
\definecolor{darkgray}{rgb}{0.41, 0.41, 0.41}
\definecolor{green}{rgb}{0.0, 0.5, 0.0}
\usepackage{listings}
\lstset{language=C++, 
	numbers=left,
	firstnumber=1,
	numberfirstline=true,
    basicstyle=\linespread{0.8}\ttfamily,
    keywordstyle=\color{blue}\ttfamily,
	showstringspaces=false,
    stringstyle=\color{red}\ttfamily,
    commentstyle=\color{green}\ttfamily,
	identifierstyle=\color{darkgray}\ttfamily,
    morecomment=[l][\color{magenta}]{\#},
	tabsize=4,
    breaklines=true
}

\begin{document}

\input{./prop_title.tex}


%%%%%%%%%%%
% SUMARIO %
%%%%%%%%%%%
\begin{abstract}
	
Our objective is to develop a Minix version of the card game Set. We will try to use all the devices, starting, of course with the mandatory ones and progressing to others as time and ability permits. This document presents the description of the game we want to implement as well as the devices to be used and their role.

\end{abstract}

\tableofcontents

%%%%%%%%%%%%%%
% INTRODUCAO %
%%%%%%%%%%%%%%
\chapter{The game Set}

Set is a card game that should be played in real time, in the sense that all players may act simultaneously. It was designed by Marsha Falco in 1974. There are 81 cards in a deck of Set, they are all distinct and have 4 varying features:
\begin{itemize}
\item color - red, green or blue;
\item symbol - diamond, oval or squiggle;
\item number - one, two or three;
\item shading - open, stripped or solid.
\end{itemize}
Each card has one of the possible combinations of these features, for example, there is a card with three striped green diamonds, and another card with one solid blue squiggle.

There are several games that can be played with these cards, but they all involve the concept of a set. A set consists of three cards such that for each feature, they all agree or they all differ. That is:
\begin{itemize}
\item they all have the same color or they all have different colors;
\item they all have the same symbol or they all have different symbols;
\item they all have the same number or they all have different numbers;
\item they all have the same shading or they all have different shadings.
\end{itemize}
For example, these three cards form a set:
\begin{itemize}
\item one red striped diamond;
\item two red solid diamonds;
\item three red open diamonds.
\end{itemize}
This implies that given any two cards from the deck, there will be exactly one other card that forms a set with them.

In the standard Set game, twelve cards are laid down. The first player that sees a set calls ``Set!'' and takes the cards in the set. Three new cards are dealt to replace those and the game proceeds. If a player makes a mistake and chooses three cards that do not form a set, they are penalized. If there is no set among the twelve cards, the dealer deals out three more cards to make fifteen dealt cards, or eighteen or more, as necessary. This process of dealing by threes and finding sets continues until the deck is exhausted and there are no more sets on the table. At this point, whoever has collected the most sets wins.


\chapter{Main features for our Minix Set}

We wish to implement the game in two main modes. The first mode, which should be simpler to implement, is a single player version. In this version, the player is alone (or in a group but playing together instead of competing). The cards are dealt as in the standard game variant and the player identifies sets (or the fact that there is no set among the dealt cards). The objective is to finish the game as fast as possible, without making mistakes. Each mistake results in a time penalty and the total time is used to compute the players score. We might also make a slight variation of this version where the player only as some time to pick each set or looses the game.

In the second version of the game, two players (in different computers connected by the serial port) compete against each other, as in the standard Set game. If the serial port implementation proves to be difficult, we may choose to implement this dual player mode in a single computer. In this variant, the players are together and use different keys on the keyboard to declare they have found a set.

In addition to the two game modes, we want to implement this software with the following features:
\begin{itemize}
\item it keeps a log of player scores;
\item it allows the user to save and load games;
\item it allows players to pause the game.
\end{itemize}

\chapter{Devices to be used}

\section{Video card in graphics mode}

We will use the video card to display the images of the cards, as well as the rest of the graphic interface features, like menus, player scores and so on.

\section{Timer}

In order to keep track of the time in single player mode, we will use the Timer in interruption mode. It will also be used if we implement the version where only some time is available for each play. Finally, we may keep track of time in dual player mode, but that is not a priority.

\section{Keyboard}

The keyboard will be used by the players whenever they find a set. In single player mode, a single specific key, like space, will be used for this effect. In two player mode, if different systems are in use, the key will also be space, but if the players are in the same system, we will use different keys on opposite sides of the keyboard, one for each player. 

After the initial key denoting that a set has been found, we may use arrow, number or even alphabet keys to choose the three cards of the set. We do not consider this a very user friendly way to do so, but it seems easier to implement than the mouse, hence its presence at least in initial versions of the game. We will use specific keys to pause, exit or save the game. Initial game setup will also be done using the keyboard. 

All these keyboard features will be implemented using interruptions.

\section{Mouse}

The mouse should be implemented as the preferred way to choose the three cards of the set. Eventually, it may also be used for initial game setup and to press pause, exit or save buttons. This will be done using interruptions.

\section{RTC}

When creating the log of player scores, in order to obtain the date and time for each log register, we should use the RTC.

\section{Serial Port}

The serial port, if we manage to implement it, is the preferred way for the two player mode of the game, connecting the two computers and keeping the game synchronized between them.

\end{document}
